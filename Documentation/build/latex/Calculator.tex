%% Generated by Sphinx.
\def\sphinxdocclass{report}
\documentclass[letterpaper,10pt,english]{sphinxmanual}
\ifdefined\pdfpxdimen
   \let\sphinxpxdimen\pdfpxdimen\else\newdimen\sphinxpxdimen
\fi \sphinxpxdimen=.75bp\relax

\usepackage[utf8]{inputenc}
\ifdefined\DeclareUnicodeCharacter
  \DeclareUnicodeCharacter{00A0}{\nobreakspace}
\fi
\usepackage{cmap}
\usepackage[T1]{fontenc}
\usepackage{amsmath,amssymb,amstext}
\usepackage{babel}
\usepackage{times}
\usepackage[Bjarne]{fncychap}
\usepackage{longtable}
\usepackage{sphinx}

\usepackage{geometry}
\usepackage{multirow}
\usepackage{eqparbox}

% Include hyperref last.
\usepackage{hyperref}
% Fix anchor placement for figures with captions.
\usepackage{hypcap}% it must be loaded after hyperref.
% Set up styles of URL: it should be placed after hyperref.
\urlstyle{same}
\addto\captionsenglish{\renewcommand{\contentsname}{Contents:}}

\addto\captionsenglish{\renewcommand{\figurename}{Fig.}}
\addto\captionsenglish{\renewcommand{\tablename}{Table}}
\addto\captionsenglish{\renewcommand{\literalblockname}{Listing}}

\addto\extrasenglish{\def\pageautorefname{page}}

\setcounter{tocdepth}{1}



\title{Calculator Documentation}
\date{Mar 14, 2017}
\release{0.1}
\author{Muhammad Yaseen\\Geetha Rajadurai\\Ali Khozravi\\Majid Zeraati\\Mortaza Jalalvand}
\newcommand{\sphinxlogo}{}
\renewcommand{\releasename}{Release}
\makeindex

\begin{document}

\maketitle
\sphinxtableofcontents
\phantomsection\label{\detokenize{index::doc}}



\chapter{Introduction}
\label{\detokenize{index:introduction}}\label{\detokenize{index:welcome-to-calculator-s-documentation}}

\section{Introduction}
\label{\detokenize{index:id1}}
This is the Introduction

\begin{sphinxVerbatim}[commandchars=\\\{\}]
\PYG{l+m+mi}{3}\PYG{o}{+}\PYG{l+m+mi}{5}
\PYG{n}{a} \PYG{o}{=} \PYG{p}{[}\PYG{l+m+mi}{2}\PYG{p}{,} \PYG{l+m+mi}{4}\PYG{p}{,} \PYG{l+m+mi}{5}\PYG{p}{,} \PYG{l+m+mi}{7}\PYG{p}{,} \PYG{l+m+mi}{4}\PYG{p}{,} \PYG{l+m+mi}{2}\PYG{p}{,} \PYG{l+m+mi}{6}\PYG{p}{]}
\PYG{n}{a}
\PYG{n}{a}\PYG{p}{[}\PYG{l+m+mi}{0}\PYG{p}{]}
\PYG{n}{a}\PYG{p}{[}\PYG{l+m+mi}{1}\PYG{p}{]}
\PYG{n}{a}\PYG{p}{[}\PYG{o}{\PYGZhy{}}\PYG{l+m+mi}{1}\PYG{p}{]}

\end{sphinxVerbatim}

another test.

\begin{sphinxVerbatim}[commandchars=\\\{\}]
\PYG{c+c1}{\PYGZsh{}include \PYGZlt{}iostream\PYGZgt{}}

\PYG{n}{using} \PYG{n}{namespace} \PYG{n}{std}\PYG{p}{;}

\PYG{n+nb}{int} \PYG{n}{main} \PYG{p}{(}\PYG{p}{)} \PYG{p}{\PYGZob{}}
  \PYG{n}{cout} \PYG{o}{\PYGZlt{}\PYGZlt{}} \PYG{l+s+s2}{\PYGZdq{}}\PYG{l+s+s2}{Hello}\PYG{l+s+s2}{\PYGZdq{}} \PYG{o}{\PYGZlt{}\PYGZlt{}} \PYG{n}{endl}\PYG{p}{;}
  \PYG{k}{return} \PYG{l+m+mi}{0}\PYG{p}{;}
\PYG{p}{\PYGZcb{}}
\end{sphinxVerbatim}

Third test.
\begin{equation*}
\begin{split}\frac{ \sum_{t=0}^{N}f(t,k) }{N}\end{split}
\end{equation*}
and inline math \(\frac{ \sum_{t=0}^{N}f(t,k) }{N}\).


\chapter{Indices and tables}
\label{\detokenize{index:indices-and-tables}}\begin{itemize}
\item {} 
\DUrole{xref,std,std-ref}{genindex}

\item {} 
\DUrole{xref,std,std-ref}{modindex}

\item {} 
\DUrole{xref,std,std-ref}{search}

\end{itemize}



\renewcommand{\indexname}{Index}
\printindex
\end{document}